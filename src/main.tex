%! suppress = Makeatletter
%! suppress = Makeatletter
\documentclass[11pt]{report}

\usepackage[T1]{fontenc}
\usepackage[utf8]{inputenc}
\usepackage{graphicx}
\usepackage{amsmath,amssymb,amsfonts}
\usepackage{polski}
\usepackage[raggedright]{titlesec}
\usepackage{indentfirst}
\usepackage{listings}
\usepackage{hyperref}
\usepackage[backend=biber, bibencoding=utf8, style=ieee, dashed=false, isbn=false, doi=false, sorting=anyvt]{biblatex}
\usepackage{caption}
\captionsetup{%
justification=raggedright,
labelfont=bf,
singlelinecheck=off
}

%\addbibresource{library.bib}
\addbibresource{NEW.bib}

\pagestyle{headings}

\renewcommand{\chaptername}{Rozdział}
\renewcommand{\contentsname}{Spis treści}
\renewcommand{\figurename}{Rys.}
\renewcommand{\tablename}{Tab.}
\renewcommand{\listfigurename}{Spis rysunków}
\renewcommand{\listtablename}{Spis tabel}
\renewcommand{\bibname}{Bibliografia}

\makeatletter
\renewcommand{\l@section}{\@dottedtocline{1}{1.5em}{2.6em}}
\renewcommand{\l@subsection}{\@dottedtocline{2}{4.0em}{3.6em}}
\renewcommand{\l@subsubsection}{\@dottedtocline{3}{7.4em}{4.5em}}
\makeatother

\begin{document}
    \begin{titlepage}
        \centering
        \includegraphics[width=\linewidth]{fig/AGH.jpg}
        \center{\scshape WYDZIAŁ INFORMATYKI, ELEKTRONIKI\\ i~TELEKOMUNIKACJI}
        \vspace{0.04\textheight}
        \center{\textbf PRACA DYPLOMOWA MAGISTERSKA}
        \vspace{0.03\textheight}
        \center{\LARGE\bfseries Agentowe środowisko do sterowania procesem identyfikacji i analizy wzorców czasowych w~notkach o zdarzeniach politycznych}
        \center{Multi-agent environment for management of the process of identifying and~analyzing time patterns in news about political events}
        \vspace{0.12\textheight}
        \begin{tabbing}
            \hspace{0.3\textwidth}\=\\
            Autor: \>Michał Patyk\\
            Kierunek studiów:\> Informatyka\\
            Opiekun pracy:\> dr hab.
            inż.
            Jarosław Koźlak
        \end{tabbing}
        \vspace{0.05\textheight}
        \center{Kraków 2021}
    \end{titlepage}

    \tableofcontents


    \chapter{Wstęp}\label{ch:wstęp}


    \chapter{Przegląd dziedziny}\label{ch:przegląd-dziedziny}
    W tym rozdziale w części~\ref{sec:gdelt} opisany został zbiór GDELT\@.
    W części~\ref{sec:przegląd-istniejących-analiz} przedstawione zostały prace dotyczące zbioru danych GDELT\@.


    \section{GDELT}\label{sec:gdelt}


    \section{Przegląd istniejących analiz}\label{sec:przegląd-istniejących-analiz}
    \subsection{Jarosz}\label{subsec:jarosz}
Celem pracy \cite{Jarosz2020} identyfikacja wzorców w GDELT oraz opracowanie i ewaluacja algorytmów oraz metod analizy wzorców z tego zbioru\@.
Badane wzorce zostały podzielone na statyczne oraz dynamiczne.
Głównymi wyszczególnionymi elementami relacji sa powtarzalność oraz intensywność.
Wzorce statyczne: sojusznik, nieprzyjaciel, symetria, asymetria, istotność, mocarstwo-klient.
Wzorce dynamiczne: stabilna relacja, jedyna zmiana, pojedyncze zaburzenie, oscylacja, wzajemność, podobny ciąg dopasowań, zależność w czasie.
Dodatkowe wzorce dynamiczne COVID-19: spadek nastrojów, nowa normalność, gratulacje, skupienie na walce.

\subsection{Skwara}\label{subsec:skwara}
W pracy Odkrywanie ukrytych informacji w mediach społecznościowych~\cite{Skwara2019} autor bada częste lub nietypowe wzorce zachowań użytkowników mediów społecznościowych.
Wzorce: wierny komentator, cięty komentator, dwaj komentatorzy, dwóch na jednego, komentator dwóch autorów, czujny komentator, prawo przechodniości.

\subsection{Rudek}\label{subsec:rudek}
W pracy \cite{10.1093/jigpal/jzaa042} autorzy identyfikują i kategoryzują wzorce opisujące interakcje między użytkownikami w sieciach społecznościowych.
Uwaga została skupiona na wzorcach w których występują częste interakcje dwóch lub więcej użytkowników oraz na pozycji społecznej tych użytkowników w sieci.
Zbiór danych został podzielony na N okien czasowych.
Lista postów była eksportowana dla każdego okna, a na jej podstawie tworzono słownik.
Przy pomocy słownika analizowano relacje miedzy użytkownikami.

\subsection{Yan}\label{subsec:yan}
W pracy \cite{Yan2012} autorzy proponują framework oparty na ważności dla charakteryzowania i wyodrębniania zmieniających sie wzorców w sieciach przedstawiających notatki prasowe.
Zdefiniowano dwa wskaźniki istotności, aby scharakteryzować ewolucję i odkryć zmiany topologii o doniosłym znaczeniu.
Cały proces znajdowania zachowań dynamicznych jest kierowany przez punktację istotności.

\subsection{inne}\label{subsec:inne}
Do analizy \cite{Buckingham2020,Levin2018,Yuan2017}.


    \section{Systemy Agentowe}

    \subsection{Gerhard Weiss}
    \begin{figure}[!htp]
        \centering
        \includegraphics[width=\linewidth]{fig/agent_środowisko_weiss.png}
        \caption{Agent w swoim środowisku. (źródło: Multiagent Systems \cite{55066420130101})}
        \label{fig:agent}
    \end{figure}


    \chapter{Narzędzia}


    \section{Mesa}
    Mesa~\cite{Masad2015} to biblioteka do modelowania agentowego w Pythonie.
    Pozwala na szybkie stworzenie modelu agentowego przy wykorzystaniu wbudowanych komponentów.
    \href{https://github.com/projectmesa/mesa}{Project~Mesa~github}
    Ostatnie zmiany na githubie 28 listopada 2020 (stan na 11XII2020r.).


    \section{Python Agent DEvelopment framework}
    Framework PADE~\cite{Melo2019} umożliwia rozwój, wykonanie i zarządzanie wieloagentowymi systemami w rozproszonym środowisku.
    \href{https://github.com/grei-ufc/pade}{Smart~Grids~Research~Group~-~UFC~github}
    Ostatnie zmiany na githubie 22 maja 2020 (stan na 11XII2020r.).


    \section{Smart Python multi-Agent Development Environment}
    Platforma wieloagentowa SPADE jest oparta o XMPP\cite{Saint-Andre2007}.
    \href{https://github.com/javipalanca/spade}{javipalanca~github}
    Ostatnie zmiany na githubie 22 maja 2020 (stan na 11XII2020r.).


    \chapter{Koncepcja}\label{ch:koncepcja}
    \input{miary}


    \chapter{Realizacja}\label{ch:realizacja}


    \chapter{Ewaluacja}\label{ch:ewaluacja}


    \chapter{Podsumowanie}\label{ch:podsumowanie}

    \inputencoding{utf8}

    \newpage
    \addcontentsline{toc}{chapter}{Bibliografia}

%    TODO remove this line
    \nocite{*}
    \printbibliography[title={Bibliografia}]


\end{document}


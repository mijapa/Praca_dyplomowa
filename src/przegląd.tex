\subsection{Jarosz}\label{subsec:jarosz}
Celem pracy \cite{Jarosz2020} identyfikacja wzorców w GDELT oraz opracowanie i ewaluacja algorytmów oraz metod analizy wzorców z tego zbioru\@.
Badane wzorce zostały podzielone na statyczne oraz dynamiczne.
Głównymi wyszczególnionymi elementami relacji sa powtarzalność oraz intensywność.
Wzorce statyczne: sojusznik, nieprzyjaciel, symetria, asymetria, istotność, mocarstwo-klient.
Wzorce dynamiczne: stabilna relacja, jedyna zmiana, pojedyncze zaburzenie, oscylacja, wzajemność, podobny ciąg dopasowań, zależność w czasie.
Dodatkowe wzorce dynamiczne COVID-19: spadek nastrojów, nowa normalność, gratulacje, skupienie na walce.

\subsection{Skwara}\label{subsec:skwara}
W pracy Odkrywanie ukrytych informacji w mediach społecznościowych~\cite{Skwara2019} autor bada częste lub nietypowe wzorce zachowań użytkowników mediów społecznościowych.
Wzorce: wierny komentator, cięty komentator, dwaj komentatorzy, dwóch na jednego, komentator dwóch autorów, czujny komentator, prawo przechodniości.

\subsection{Rudek}\label{subsec:rudek}
W pracy \cite{10.1093/jigpal/jzaa042} autorzy identyfikują i kategoryzują wzorce opisujące interakcje między użytkownikami w sieciach społecznościowych.
Uwaga została skupiona na wzorcach w których występują częste interakcje dwóch lub więcej użytkowników oraz na pozycji społecznej tych użytkowników w sieci.
Zbiór danych został podzielony na N okien czasowych.
Lista postów była eksportowana dla każdego okna, a na jej podstawie tworzono słownik.
Przy pomocy słownika analizowano relacje miedzy użytkownikami.

\subsection{Yan}\label{subsec:yan}
W pracy \cite{Yan2012} autorzy proponują framework oparty na ważności dla charakteryzowania i wyodrębniania zmieniających sie wzorców w sieciach przedstawiających notatki prasowe.
Zdefiniowano dwa wskaźniki istotności, aby scharakteryzować ewolucję i odkryć zmiany topologii o doniosłym znaczeniu.
Cały proces znajdowania zachowań dynamicznych jest kierowany przez punktację istotności.

\subsection{inne}\label{subsec:inne}
Do analizy \cite{Buckingham2020,Levin2018,Yuan2017}.